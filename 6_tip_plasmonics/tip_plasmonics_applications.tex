\documentclass{article}
\usepackage{import}
\subimport{../}{preamble}
\begin{document}

\section{Improved Field Enhancement of Spherical Au Nanoparticle Tips}
\label{sec:tip_applications}

A result of spherical Au tips sustaining radiative LSPs is that their plasmonic contribution to the field enhancement can outperform the lightning rod contribution in sharp tips (assuming no near-field plasmonic excitation in sharp tips). The field enhancements for both sharp and spherical Au tips, more specifically AuNP-on-Pt tips, are determined in a side illumination configuration using Raman scattering.

\begin{figure}[bt]
\centering
\def\svgwidth{0.6\textwidth}\fontsize{10pt}{1em}\selectfont
\subimport{./figures/}{ters_setup.pdf_tex}
\caption[Experimental geometry for dark-field spectroscopy and SERS measurements]{\textbf{Experimental geometry for dark-field spectroscopy and SERS measurements.} A \SI{250}{nm} diameter spherical AuNP grown onto a Pt-coated AFM tip is spectroscopically studied using a supercontinuum laser in a dark-field configuration (a). The tip is then brought within \SI{1}{nm} of a benzenethiol-coated sharp Au tip under \SI{638}{nm} illumination to measure SERS spectra (b).}
\label{fig:ters_setup}
\end{figure}

% Introduction to the method and experiment configuration
SDF spectroscopy is used in conjunction with Raman spectroscopy in a modified version of the microscope platform, enabling both techniques (though not simultaneously). Tips are mounted opposite a second benzenethiol-coated sharp Au AFM tip in a tip-to-tip configuration, mimicking a plasmonic bow-tie antenna (\figurename~\ref{fig:ters_setup}). This configuration is used to obtain good optical access to the intertip gap for spectroscopically probing its plasmonic properties. Benzenethiol (BTh) is used as a Raman marker for measuring the relative field enhancement of AuNP tips due to its strong Raman response and well-known spectra \cite{mahajan2009, dudin2010}.
% BTh coating method
BTh (VWR International Thiophenol for synthesis) is diluted to \SI{5}{mM} solutions in ethanol (Sigma-Aldrich). A standard Au-coated AFM tip, for use as a SERS substrate, is prepared by coating a monolayer of BTh onto the surface. This is achieved by submerging it in \SI{100}{mM} ethanolic BTh solution for \SI{1}{\minute} followed by rinsing with ethanol and drying in nitrogen. This is repeated 5 times to ensure complete monolayer coverage. Tips used as plasmonic probes are not coated in BTh.

% Experimental TERS/characterisation procedure
With the BTh tip retracted, a SDF scattering spectrum of the enhancing tip apex is acquired. After characterisation, the microscope optics are modified into a TERS configuration and the enhancing tip is aligned to the BTh tip using the capacitive alignment technique described in chapter~\ref{sec:tip_alignment}.%
\footnote{The optics are modified in the sense that the laser input is switched, filters are inserted and the dark-field iris is opened.}
Once aligned, the gap size is reduced to $\sim$\SI{1}{nm}, limited by the thickness of the assembled BTh molecular layer, and illuminated through a $100\times$ 0.9\,NA visible objective with \SI{3}{mW} (\SI{1.9}{\mega\watt\per\centi\metre\squared}) of \SI{638}{nm} laser light incident on the gap, polarised along the tip axis. Scattered light is collected through the same objective and confocally localised. Raman spectra are filtered using a \SI{650}{nm} long-pass filter (Chroma) prior to dispersion in a spectrometer. Contact dynamics, measured using AFM, confirm that tips come into physical contact while separated by a BTh layer.

% Simulations
Near-field calculations for the spherical Au tip are computed for comparison with experimental results and to understand the enhancement mechanism. The near-field distribution at \SI{633}{nm} and the spectrum \SI{1}{nm} from the apex are calculated using the full electrodynamic boundary-element method \cite{deabajo1997, deabajo2002}.%
\footnote{Near-field calculations carried out by L.\,O.\ Herrmann.}
The spherical tip is modelled as a Pt cone with half-angle \SI{20}{\degree} with a \SI{250}{nm} diameter AuNP attached to its end. The neck diameter between sphere and tip is \SI{100}{nm}. The tip is illuminated with a plane wave polarised along the tip axis.

\begin{figure}[bt]
\centering
\begin{tikzpicture}
\node [below right] at (0,0) {\includegraphics{figures/ters_comparison}};
\node [right] at (0.3,-0.2) {\textbf{a}};
\node [right] at (0.3,-5.0) {\textbf{b}};
\node [below right] at (9.5,-0.1) {\includegraphics{figures/aunp_tip_nearfield}};
\node at (9.9,-0.2) {\textbf{c}};
\end{tikzpicture}
\caption[Application of sharp Au and AuNP-on-Pt tips to enhancing Raman scattering]{\textbf{Application of sharp Au and AuNP-on-Pt tips to enhancing Raman scattering.} (a,b) Comparative TERS and dark-field spectroscopy of sharp Au and AuNP tips. (a) Tip-enhanced Raman spectra of a benzenethiol-coated Au AFM probe brought close to the AuNP tip (red), compared to a sharp Au AFM tip (blue). (b) Dark-field optical scattering of AuNP (red) and sharp Au (blue) AFM tips, with calculated relative intensity enhancement \SI{0.5}{nm} from the AuNP tip apex (dashed). The inset shows an SEM image of the \SI{250}{nm} AuNP tip. (c) Calculated field enhancement profile for a \SI{250}{nm} diameter AuNP at the end of a \SI{1500}{nm} long Pt tip. The neck join is \SI{50}{nm} wide and the tip is under longitudinally polarised plane wave illumination at \SI{633}{nm}.}
\label{fig:ters_comparison}
\end{figure}

A \SI{250}{nm} diameter spherical AuNP-on-Pt tip, grown as described in chapter \ref{sec:initial_fabrication} (\SI{-8}{V}, \SI{150}{ms} exposure), is used to demonstrate the augmented plasmonic properties of spherically nanostructured tips. Raman spectra of BTh molecules in the tip dimer gap are greatly enhanced by 30$\times$ when using a AuNP tip in place of a sharp Au tip (\figurename~\ref{fig:ters_comparison}a). As the same spectrometer is used for both broadband scattering spectra and SERS spectra, its restricted spectral resolution (300--\SI{1100}{nm} bandwidth), combined with the relatively broad laser diode linewidth, blurs the characteristic multiple Raman peaks of BTh between 1000--\SI{1600}{\per\centi\metre}. However the resulting observation of two broad peaks in this region affirms the presence of BTh in the gap between tips. Absence of an S-H peak around \SI{2200}{\per\centi\metre} suggests good monolayer coverage. The background signal is also enhanced across a broad bandwidth, as is typical for SERS around a plasmon resonance \cite{mahajan2009}.

SDF scattering spectra (\figurename~\ref{fig:ters_comparison}b), taken of individual tips prior to SERS measurements, show that the increased Raman enhancement when using a AuNP tip is due to resonant excitation of a LSP around \SI{630}{nm}, not present in sharp Au tips. This is in good agreement with boundary element calculations of the near-field enhancement at the AuNP tip apex with a visible plasmon resonance observed across the AuNP (\figurename~\ref{fig:ters_comparison}b,c).
Coupling between this LSP in the AuNP tip with a BTh-coated sharp Au tip forms a confined gap plasmon mode. Since coupling is between higher order modes in the sharp Au tip, shifting of this resonance as a function of gap size is weak \cite{downes2006, hugall2012}. A relative SERS enhancement is estimated by taking into account the confinement and mode volume of a LSP in the gap in each case.

% Field enhancement calculations
LSP mode volumes are estimated using a cylindrical gap mode model. The lateral width of a gap plasmon mode is calculated using $w=\sqrt{R_{\mathrm{eff}}d}$, where $R_{\mathrm{eff}}$ is the effective radius of the particles, $\sqrt{R_1R_2}$, comprising the plasmonic dimer and $d$ is the width of the gap separating particles \cite{romero2006}.
\footnote{$w=\sqrt{Rd}$ is mentioned in the main text but not derived.}
%\footnote{\color{red}The use of $R_{\mathrm{eff}}=\sqrt{R_1 R_2}$ is justified by...}
This results in lateral mode widths of \SI{4.5}{nm} for the sharp Au tip of \SI{20}{nm} radius and \SI{7.1}{nm} for the \SI{125}{nm} radius AuNP tip. Assuming a cylindrical gap mode ($V=\pi w^2d/4$) yields mode volumes of \SI{15.7}{nm\cubed} and \SI{39.3}{nm\cubed}, respectively. These define the near-field contribution to Raman scattering and a relative field enhancement is obtained using,
\begin{equation}
	\mathit{FE}_{\mathrm{rel}} = \frac{N_{\mathrm{AuNP}} / V_{\mathrm{AuNP}}}{N_{\mathrm{tip}} / V_{\mathrm{tip}}}
\end{equation}
where $N$ is the Raman signal counts and $V$ is the mode volume. This evaluates to a relative SERS enhancement of 12.
Since the LSP is laterally confined to only \SI{7}{nm} within this gap the enhanced Raman signal is the result of scattering contributions from only a very small number of molecules. Lower limit absolute Raman enhancements are estimated using,
\begin{equation}
	\mathit{FE}_{\mathrm{abs}} = \frac{N_{nf} / V_{nf}}{N_{ff} / V_{ff}}
\end{equation}
where $N_{ff}$, the number of counts obtained using only far-field laser light, is assumed to be \SI{0.1}{counts.s^{-1}.mW^{-1}} from the noise levels since signals are below the signal to noise level and $V_{ff}$ is assumed to be \SI{25000}{nm\cubed} based upon the surface of a conical tip exposed to the focal volume of a diffraction limited spot ($d = \SI{412}{nm}$ at $\lambda = \SI{638}{nm}$). % How does this calculation compare with the actual measured spot size? Doesn't matter since SERS beam is Gaussian and formula is accurate enough.
This expression yields absolute, lower-bound Raman enhancements of \num{1.9e5} for a \SI{250}{nm} AuNP tip and \num{1.6e4} for a sharp Au tip. Though absolute estimates are not as high as the expected \num{e7}--\num{e8} enhancements reported in the literature \cite{pettinger2012}, the relative SERS enhancement observed with the AuNP tip is indeed comparable to previously reported results \cite{umakoshi2012}.

% Conclusions of plasmonic tips
These optical measurements confirm that AuNP tips provide increased field enhancement compared to sharp Au tips due to a strong LSP excitation. Lack of any strong peaks around \SI{600}{nm} in dark-field spectra of sharp Au tips suggests that any plasmons present are weakly coupled and do not scatter strongly in this illumination geometry, resulting in a lower observed field enhancement. Though the dual lightning rod contribution in the sharp tip dimer could provide a larger potential enhancement, failure to couple light to the near-field prevents access to this enhancement. On the other hand, AuNP tips are well suited to high enhancements when illuminated at the appropriate plasmonic resonances, behaving as optical antennae.

Whilst a number of plasmonic probes have been developed recently, several useful features are obtained here. By using standard AFM probes as a basis, these AuNP tips maintain their functionality as AFM probes for force microscopy. The metallic coating of these tips also allows for simultaneous electrical measurements whilst performing optical and AFM force measurements. These tips therefore function as standard electrical AFM probes with added plasmonic functionality.
Furthermore, such tips also show excellent resistance to damage at the tip apex after multiple surface contacts, though surfaces do become deformed after heavy use. Their robust nature is attributed to the direct growth of the AuNP root across the pyramidal tip end. This is a significant improvement over currently-available commercial spherical AFM tips, in which spheres break from the tip and adhere to the contact surface after only one or a small number of contact cycles. All of this is beneficial for creating a dynamically controllable plasmonic dimer on which to perform measurements in the quantum regime.
Further applications of spherical metallic tips can be envisaged, for instance in plasmonic optical trapping \cite{lindquist2013} because the tips in the present geometry can conveniently act as a heat sink reducing the problematic optical heating observed, and resulting thermal damage.

\end{document}