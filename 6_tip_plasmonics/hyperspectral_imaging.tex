\documentclass{article}
\usepackage{import}
\subimport{../}{preamble}
\begin{document}

\section{Optical Characterisation of Nanostructures using Hyperspectral Imaging}
\label{sec:hyperspectral_imaging}

% Introduction to hyperspectral imaging
Hyperspectral imaging encompasses a range of optical techniques which image using pixels comprised of a spectrum rather than an RGB colour value. This is advantageous over regular imaging as more quantitative information can be extracted from an image. Hence hyperspectral imaging techniques have become commonplace across many diverse fields, including microscopy \cite{schultz2001hyperspectral, leavesley2012hyperspectral}, astrophysics \cite{hege2004hyperspectral}, remote sensing and geology \cite{hackwell1996lwir, shaw2003spectral}, food standards \cite{kim2001hyperspectral, gowen2007hyperspectral}, and medical imaging \cite{vo2004hyperspectral, martin2006development, lu2014medical}. Within each of these fields, the features in an image are more clearly identified by their spectral signatures. In this instance, scanning confocal hyperspectral imaging is applied to optically characterise both sharp and nanostructured metallic tips and identify optical resonances originating from \gls{lsp} excitation.%
\footnote{This technique has also been applied to other periodic, extended nanostructures that are not discussed within the context of this thesis.} % Lee suggested this

\begin{figure}[bt]
\centering
\fontsize{10pt}{1em}\selectfont
\def\svgwidth{0.6\textwidth}
\subimport{./figures/}{simple_optics_layout.pdf_tex}
\caption[Experiment configuration for hyperspectral imaging]{\textbf{Experiment configuration for hyperspectral imaging.} The laser is centered on the tip apex for imaging. The tip is scanned across the beam in a grid with spectra acquired at each position. The resulting image then contains 1044 colours at each pixel instead of the usual 3 (RGB).}
\label{fig:simple_optics_layout}
\end{figure}

Scanning confocal hyperspectral imaging falls under the category of spatially scanned imaging. Tips are scanned in a grid under the laser spot and the spectral content of the confocal sampling volume is measured at each point using a spectrometer instead of a photodiode or CCD. Images are then formed at a given wavelength or across a wavelength band. In this instance, using a spectrometer, each image pixel is digitised into 1044 bins between 400--\SI{1200}{nm} rather than the conventional 3 RGB colour bands.%
\footnote{The \SI{0.8}{nm} wavelength resolution of spectra classifies this procedure as hyperspectral, as opposed to multi-spectral imaging in which images are formed using fewer, much broader, wavelength bands.}
This approach to hyperspectral imaging has previously been used to identify distributed plasmon modes in aggregated AuNP colloids \cite{herrmann2013} and to image SPPs \cite{bashevoy2007hyperspectral}. By using this technique in the current microscope configuration, as shown in \figurename~\ref{fig:simple_optics_layout}, LSPs can be spatially identified with sub-diffraction-limited resolutions around \SI{250}{nm}. Combining it with \SI{80}{nm} AuNPs also enabled measurement of the microscope PSF and chromatic aberrations, as discussed in chapter \ref{sec:optical_design}.

Fast image acquisition is made possible by utilising the ultra-high brightness supercontinuum laser source and sensitive, TE-cooled, benchtop spectrometers with 10--\SI{20}{ms} integration times. Image acquisition is limited only by the integration time at each pixel and the $\sim$\SI{30}{ms} movement time between pixels. The focal intensity is, on average, $\sim$\SI{e5}{\watt\per\centi\metre\squared}, which is below the damage threshold for \SI{50}{nm} metallic tip coatings. The illumination and collection configuration is fixed using a preset reference intensity between different samples in order to more quantitatively compare images. Measured spectra of metallic nanostructures are normalised to a BF reflection spectrum of the same flat metal to show structural effects only, such as plasmons. %By using this approach spectral changes between the apex of a tip and its bulk surfaces can be determined.

While not the fastest or most advanced method of acquiring hyperspectral images, spatial scanning is made efficient when used with a supercontinuum white-light source and images benefit from confocal localisation. Other imaging techniques, categorised under ``spectral scanning", ``non-scanning" and ``spatio-spectral scanning", have been developed to more efficiently produce hyperspectral images under specific conditions.
% Spectral scanning
Spectral scanning involves wide-field imaging through either a range of bandpass filters \cite{iga2012development}, a tuneable liquid crystal filter \cite{slawson1999hyperspectral, gat2000imaging} or an etalon \cite{daly2000tunable}, which is appropriate if studying large areas or if confocal localisation is not required. Similarly, if optical sectioning is not necessary, samples can be imaged with a single direction line scan using an imaging spectrograph (monochromator with CCD) rather than using a two-dimensional grid scan \cite{schultz2001hyperspectral}.
% Non-scanning
Non-scanning or snapshot techniques are more complex and computationally demanding than scanning techniques as both spatial and spectral information are acquired in a single static measurement. Imaging is typically achieving using a computed tomography imaging spectrometer (CTIS) \cite{okamoto1991simultaneous, bulygin1992spectrotomography, okamoto1993simultaneous, descour1995computed}, in which a 2D dispersive grating placed in the Fourier plane results in spectrally separated images on the CCD.
% Spatio-spectral scanning
Spatio-spectro scanning is the most recent technique, developed in 2014, and involves diagonally scanning through the sample data cube where each point along an axis in the spatial image has a different wavelength \cite{grusche2014basic}. % images would then need to be translated across the spectral range

% Why we still use spatial scanning
Despite the potential improvements gained by using more advanced hyperspectral imaging techniques, spatial point scanning is deemed the most appropriate solution for tip characterisation, if only for simplicity and compatibility with dual-tip gap spectroscopy. Scan areas are typically small, benefit from confocal localisation, and image acquisition is not time-constrained since the microscope platform is stable.

\end{document}