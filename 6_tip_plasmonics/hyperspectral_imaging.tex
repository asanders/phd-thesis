\documentclass{article}
\usepackage{import}
\subimport{../}{preamble}
\begin{document}

\section{Optical Characterisation of Nanostructures using Hyperspectral Imaging}
\label{sec:hyperspectral_imaging}

% Introduction to hyperspectral imaging
Hyperspectral imaging encompasses a range of optical techniques in which images with each pixel comprising of a spectrum rather than a RGB colour value. This is advantageous over regular imaging as more quantitative information can be extracted from an image to the extent that hyperspectral imaging techniques have become commonplace in many widely spread fields, including microscopy \cite{schultz2001hyperspectral, leavesley2012hyperspectral}, astrophysics \cite{hege2004hyperspectral}, remote sensing and geology \cite{hackwell1996lwir, shaw2003spectral}, food standards \cite{kim2001hyperspectral, gowen2007hyperspectral}, and medical imaging \cite{vo2004hyperspectral, martin2006development, lu2014medical}. Within each of these fields, the features in an image are more clearly identified by their spectral signatures. In this instance, scanning confocal hyperspectral imaging is applied to optically characterise both sharp and nanostructured metallic tips and identify \glspl{spr} originating from \gls{lsp} excitation.%
\footnote{This technique has also been applied to other periodic, extended nanostructures that are not discussed within the context of this thesis.} % Lee suggested this

\begin{figure}[bt]
\centering
\fontsize{10pt}{1em}\selectfont
\def\svgwidth{0.6\textwidth}
\subimport{./figures/}{simple_optics_layout.pdf_tex}
\caption[Experiment configuration for hyperspectral imaging]{\textbf{Experiment configuration for hyperspectral imaging.} The laser is centered on the tip apex for imaging. The tip is scanned across the beam in a grid with spectra acquired at each position. The resulting image then contains 1044 colours at each pixel instead of the usual 3 (RGB).}
\label{fig:simple_optics_layout}
\end{figure}

Scanning confocal hyperspectral imaging falls under the category of spatially scanned imaging. Tips are scanned in a grid under the laser spot and the spectral content of the confocal sampling volume is measured at each point using a spectrometer instead of a photodiode or CCD. Images are then formed at a given wavelength or across a wavelength band. In this instance, using a bench-top spectrometer, each image pixel is equally digitised into 1044 bins between 400--\SI{1200}{nm} rather than the conventional 3 RGB colour bands. The \SI{0.8}{nm} wavelength resolution of spectra classifies this procedure as hyperspectral, as opposed to multi-spectral imaging in which images are formed using fewer, much broader, wavelength bands. % citation?
This approach to hyperspectral imaging has previously been used to identify distributed plasmon modes in aggregated AuNP colloids \cite{herrmann2013} and to image SPPs \cite{bashevoy2007hyperspectral}. By using this technique in the current microscope configuration, as shown in \figurename~\ref{fig:simple_optics_layout}, LSPs can be spatially identified with sub-diffraction-limited resolutions around 250--\SI{300}{nm}. Combining it with \SI{80}{nm} AuNPs also enabled measurement of the microscope PSF and chromatic aberrations, as discussed in chapter \ref{sec:optical_design}.

Fast image acquisition is made possible by utilising the ultra-high brightness supercontinuum laser source and sensitive, TE-cooled, bench-top spectrometers with 10--\SI{20}{ms} integration times. Image acquisition is limited only by the integration time at each pixel and the $\sim$\SI{30}{ms} movement time between pixels. The focal intensity is, on average, $\sim$\SI{e8}{\milli\watt\per\centi\metre\squared}, which is below the damage threshold for \SI{50}{nm} metallic tip coatings. The illumination and collection configuration is fixed using a preset reference intensity between different samples in order to more quantitatively compare images. Measured spectra of metallic nanostructures are normalised to a spectrum of the same flat metal to show structural effects only, such as plasmons. %By using this approach spectral changes between the apex of a tip and its bulk surfaces can be determined.

While not the fastest or most advanced method of acquiring hyperspectral images, spatial scanning is made efficient when used with a supercontinuum white-light source. Other imaging techniques, categorised under ``spectral scanning", ``non-scanning" and ``spatio-spectral scanning", have been developed to more efficiently produce hyperspectral images under specific conditions.
% Spectral scanning
Spectral scanning involves wide-field imaging through either a range of bandpass filters \cite{iga2012development}, a tuneable liquid crystal filter \cite{slawson1999hyperspectral, gat2000imaging} or an etalon \cite{daly2000tunable}, which is appropriate if studying large areas or if confocal localisation is not required. Similarly, if the benefits of optical sectioning are not necessary, a single direction line scan over a sample can be performed with an imaging spectrograph (monochromator with CCD) to form a hyperspectral image rather than use a two-dimensional grid scan with single pixel acquisition \cite{schultz2001hyperspectral}.

% Non-scanning
Non-scanning or snapshot hyperspectral imaging techniques are more complex than scanning techinques as both spatial and spectral information are acquired in a single measurement without any scanning or dynamic filtering. The main method of achieving this is by using a computed tomography imaging spectrometer (CTIS) \cite{okamoto1991simultaneous, bulygin1992spectrotomography, okamoto1993simultaneous, descour1995computed}, in which a 2D dispersive grating placed in the Fourier plane splits an image into many separate spectral images projected onto a CCD. Advantages of this approach are much shorter acquisition times but necessitates a higher computational requirement to disentangle the 2D image into a cube with dimensions $(x,y,\lambda)$.
% Spatio-spectral scanning
Spatio-spectro scanning is the most recent technique, developed in 2014, and involves diagonally scanning through the sample data cube where each point along an axis in the spatial image has a different wavelength \cite{grusche2014basic}. % images would then need to be translated across the spectral range

% Why we still use spatial scanning
Despite the potential improvements gained by using more advanced hyperspectral imaging techniques, spatial point scanning is deemed the most appropriate solution for tip characterisation, if only for simplicity and compatibility with dual-tip gap spectroscopy. Image acquisition is not time-constrained since the microscope platform is stable, resulting in minimal artefacts due to sample motion, and the use of confocal imaging benefits image quality. Portable bench-top spectrometers are already incorporated into the microscope for use in other experiments and are readily accessible, therefore adapting the microscope to enable use of an imaging spectrograph and line scanning would be inconvenient. Spectral scanning is not beneficial at the current magnification due to the relatively small area occupied by the sample in the wide-field image and due to the far superior spectral resolution of spectrometers when compared to imaging through bandpass filters. For these reasons, spatial point scanning is used for characterisation.

% Note that colour reconstruction of data cubes is not covered in the thesis since it has not beed used.

\end{document}