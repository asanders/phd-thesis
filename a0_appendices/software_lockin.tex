\documentclass{article}
\usepackage{standalone} \standalonetrue
\usepackage{import}
\subimport{../}{preamble}
\begin{document}

\section{Software Lock-In Derivation}

To lock into only the signal component at the reference frequency $\omega_r$ a reference wave needs to be computed. The first step in the lock-in process is to mathematically construct a single frequency waveform at the correct harmonic using the supplied reference signal. The reference signal is typically of the form $A\sin(\omega_{rs} t + \phi_r)$, but the algorithm will also work with any periodic function since it triggers off a rising position edge.
When $\sin(\theta)=0$ and the gradient is positive ($\cos(\theta)=1$) $\theta=2n\pi$. Hence the rising edge trigger points $t_i$ occur at,
\begin{equation} \theta = \omega_{rs} t_i + \phi_r = 2n\pi. \end{equation}
Trigger times are fitted against the number of triggers (number of periods) since the start of the signal using,
\begin{equation}
t_i = \frac{1}{\omega_{rs}}(2n\pi - \phi_r) = \frac{2\pi}{\omega_{rs}}n - \frac{\phi_r}{\omega_{rs}}.
\end{equation}
A complex reference wave of the form $e^{ih(\omega_{rs} t + \phi_r)}$, where $h$ is the harmonic of the reference frequency $\omega_{rs}$ required to lock into the frequency $\omega_r$, is constructed from the $t_i=mn+c$ fit using,
\begin{align}
\omega_{rs} &= \frac{2\pi}{m}, \\
\phi_r &= \frac{mc}{2\pi}.
\end{align}
The frequency component of the signal at $\omega_r=h\omega_{rs}$ can be extracted using Fourier analysis,
\begin{equation}
Z_s(\omega_r) = \frac{2}{t} \int_0^t{Z_s(t) e^{-ih(\omega_{rs} t + \phi_r)} dt}.
\end{equation}
Discretising this not a programmable form results in,
\begin{equation}
Z_s(\omega_r) = \frac{2}{n} \sum_0^n{Z_s(t_n) e^{-ih(\omega_{rs} t_n + \phi_r)}}
\end{equation}
where $\real{Z_s(\omega_r)}$ and $\imag{Z_s(\omega_r)}$ are the $x$ and $y$ of the signal component at $\omega_r$, respectively. Polar coordinates of amplitude and phase are retrieved using the coordinate transforms,
\begin{align}
r = \sqrt{x^2 + y^2}, \\
\phi = \tan^{-1}(y/x).
\end{align}

\end{document}