% !TEX TS-program = pdflatexmk
% !BIB TS-program = bibtex

\documentclass[12pt, a4paper, oneside]{book}
\usepackage{import}
\subimport{../}{preamble}
\ExecuteBibliographyOptions{articletitle=false}
\standalonetrue
\onehalfspacing

\begin{document}
%\begin{singlespace}

\section*{Declaration}
The work presented in this thesis was carried out at the Nanophotonics Group in the Cavendish Laboratory, University of Cambridge between October 2011 and June 2015. This dissertation is the result of my own work and includes nothing which is the outcome of work done in collaboration except where specifically indicated in the text. It has not been submitted in whole or in part for any degree at this or any other university, and is less than sixty thousand words long.

{\flushright\emph{Alan Sanders}\par}

\section*{List of Talks, Posters and Publications}
\subsection*{Publications}
\begin{itemize}
\item \fullcite{sanders2014p}
\item \fullcite{benz2014p}
\mybibexclude{sanders2014p}
\mybibexclude{benz2014p}
%\item Understanding tip plasmonics
%\item Plasmon coupling
\end{itemize}

\subsection*{Conference Presentations}
\begin{itemize}
\item \fullcite{sanders2015}
\item \fullcite{sanders2014a}
\item \fullcite{sanders2013b}
\item \fullcite{sanders2013a}
\item \fullcite{sanders2012}
\mybibexclude{sanders2015}
\mybibexclude{sanders2014a}
\mybibexclude{sanders2013b}
\mybibexclude{sanders2013a}
\mybibexclude{sanders2012}
\end{itemize}

\newpage
\section*{Acknowledgements}

This project would not have been a success without the support of countless people and financial bodies over the last four years.
% Jeremy
I would first and foremost like to thank Prof.\ Jeremy Baumberg for giving me the opportunity to undertake this project and for supporting me throughout. His advice and enthusiasm kept this project on course while his wealth of knowledge and ideas meant that there was always something to work on and never a dull moment.

% Matt, Richard and Liwu
This PhD project has benefitted from the support of many talented researchers over the years, without whom the work would not have been as much of a success. I have been privileged enough to work closely with three exceptional experimental researchers: Dr.\ Matthew Hawkeye, Dr.\ Richard Bowman and Dr.\ Liwu Zhang. Their expertise and advice has been invaluable and made the experiments presented in this thesis possible. Dr.\ Matthew Hawkeye introduced me to the on-going tips project, and helped me gain proficiency in electronics. Without Dr.\ Richard Bowman, and his knowledge of optics and mechanics, the experiment would not be what it is today. From him I have gained new insights into the principles of optics and optical design. I am finally indebted to Dr.\ Liwu Zhang, under whose supervision I ventured into electrochemistry. His knowledge of electrochemistry was essential while providing constant support and interest in all work.

% People who did supplementary work for this project
I would like to thank Lars Herrmann and Daniel Sigle for providing numerical simulations to support of my experimental investigations, and Anna Lombardi and Lee Weller for performing supplementary tip measurements using their newly developed laser systems. I am also grateful to Prof.\ Javier Aizpurua and Dr.\ Ruben Esteban for their helpful discussions at each of the conferences I attended.

% Support Staff
This work has also been made easier by both the Nanophotonics and Cavendish Laboratory support staff. I would like to thank Anthony Barnett for his help in designing mechanical parts for my work and for their prompt machining, Colin Edwards for keeping all our computer systems running, and Angela Campbell for processing my constant stream of orders. Thanks go to Chris Summerfield, Gary Large, Kevin Mott and, Nigel Palfrey for their machining of the microscope used to conduct all optical work and to Barry Shores and Huw Prytherch for their assistance in designing the integrated electronics. I am grateful to EPSRC for funding this project and to NanoTools GmbH, for their cooperation in providing spherical Au tips for the experiments discussed in this thesis.

%Rest of the group
I would finally like to thank the rest of the Nanophotonics group, namely Matt Millyard, Jan Mertens, Laura Brooks, Claire Blejean, Vladimir Turek, Bart de Nijs, Felix Benz, Giuliana Di Martino, Hamid Ohadi, Will Deacon, Marie-Elena Kleemann, and Jago Del-Valle-Inclan-Redondo, who have provided many discussions, and kept me sane.

\newpage
\section*{Abstract}

Plasmonics, the confinement of light to nanometric dimensions in the form of optically-driven collection oscillations of conduction electrons, enables strong, local field enhancements, which can be exploited to realise nano-optics and nano-spectroscopy. However, the onset of quantum mechanical effects serves as a fundamental limit to plasmonic confinement in what has recently become known as the quantum regime of plasmonics.

In the present work, a dual AFM tip approach to form sub-nm plasmonic cavities is adopted to investigate the quantum regime of plasmonics and to determine, in particular, the relationship between conductance and plasmonics, a theme of great interest in the field. The technology required to reliably form sub-nm plasmonic cavities between AFM tips is further developed. A custom optical microscope with an ultra-stable nanopositioning platform has been entirely designed and optimised to facilitate experiments. Light scattering from both single and gap-coupled nanostructures over a broad wavelength range can be measured using a novel dark-field spectroscopy technique utilising a supercontinuum white-light laser.

This experimental system has been exploited to fully characterise the optical response of both commercially available sharp and spherically nanostructured Au AFM tips in order to understand their plasmonic properties. Additional spherical Au nanoparticle-tipped AFM probes are fabricated using apex-selective pulsed electrodeposition to demonstrate a simple method for introducing localised surface plasmons into a robust tip geometry. Scanning confocal hyperspectral imaging is utilised to characterise the scattering response of single nanostructures and identify localised surface plasmons. Spherical Au tips are found to exhibit an antenna-like plasmon resonance between 600--\SI{700}{nm}, not present in sharp Au tips, as a result of their nanoparticle-like apex geometry. Exciting on resonance with this plasmon improves {\color{red}Raman scattering efficiency} by a factor of 30$\times$ compared with sharp Au tips. This both demonstrates the benefits of controlled nanostructuring and why optical characterisation techniques should be adopted to supplement tip-based optical microscopy.

Finally, plasmonic interactions between two AFM tips are studied and the transition between coupled and charge transfer plasmons is dynamically observed. Simultaneous measurement of the d.c.\ current, applied force and optical scattering as tips come together is used to determine the effects of an optical conductance in a plasmonic nano-gap. Critical conductances are experimentally identified for the first time, determining the points at which quantum tunnelling and ballistic transport begin to influence plasmon coupling. This is a step towards fully understanding the relationship between conduction and plasmonics and the fundamental, quantum mechanical limitations of conventional plasmonic coupling.
%\end{singlespace}

\end{document}