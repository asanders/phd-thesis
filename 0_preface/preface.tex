% !TEX TS-program = pdflatexmk
% !BIB TS-program = bibtex

\documentclass[12pt, a4paper, oneside]{book}
\usepackage{import}
\subimport{../}{preamble}
\ExecuteBibliographyOptions{articletitle=false}
\standalonetrue
\onehalfspacing

\begin{document}
%\begin{singlespace}

\section*{Declaration}
The work presented in this thesis was carried out at the Nanophotonics Group in the Cavendish Laboratory, University of Cambridge between October 2011 and July 2015. This dissertation is the result of my own work and includes nothing which is the outcome of work done in collaboration except where specifically indicated in the text. It has not been submitted in whole or in part for any degree at this or any other university, and is less than sixty thousand words long.

{\flushright\emph{Alan Sanders}\par}

\section*{List of Talks, Posters and Publications}
\subsection*{Publications}
\begin{itemize}
\item \fullcite{sanders2014p}
\item \fullcite{benz2014p}
\mybibexclude{sanders2014p}
\mybibexclude{benz2014p}
%\item Understanding tip plasmonics
%\item Plasmon coupling
\end{itemize}

\subsection*{Conference Presentations}
\begin{itemize}
\item \fullcite{sanders2015}
\item \fullcite{sanders2014a}
\item \fullcite{sanders2013b}
\item \fullcite{sanders2013a}
\item \fullcite{sanders2012}
\mybibexclude{sanders2015}
\mybibexclude{sanders2014a}
\mybibexclude{sanders2013b}
\mybibexclude{sanders2013a}
\mybibexclude{sanders2012}
\end{itemize}

\newpage
\section*{Acknowledgements}

This project would not have been a success without the support of countless people and financial bodies over the last four years.
% Jeremy
I would first and foremost like to thank Prof.\ Jeremy Baumberg for giving me the opportunity to undertake this project and for supporting me throughout. His guidance, enthusiasm, wealth of knowledge and endless supply of ideas kept me on target and made this project what it is today.

% Matt, Richard and Liwu
I have also greatly benefitted from the support of many talented researchers over the years, without whom the work could not have been achieved. In particular, I am grateful to three post-docs who became heavily involved in the tips project and whose expertise and advise has been invaluable. I am thankful to Dr.\ Matthew Hawkeye for introducing me to the on-going tips project and helping me gain proficiency in electronics and plasmonics. I would like to thank Dr.\ Richard Bowman for giving me new insights into the principles of optics and optomechanical design and for assisting in the design and construction of the custom microscope platform. I am finally indebted to Dr.\ Liwu Zhang, under whose supervision I ventured into electrochemistry.

% People who did supplementary work for this project
I would like to thank Lars Herrmann and Daniel Sigle for providing numerical simulations to support my experimental claims, and Anna Lombardi and Lee Weller for performing supplementary tip measurements using their newly developed tuneable laser systems. I am also grateful to Prof.\ Javier Aizpurua and Dr.\ Ruben Esteban for their helpful discussions whilst attending conferences.

% Support Staff
My work has also been supported by both the Nanophotonics and Cavendish Laboratory support staff. I would like to thank Anthony Barnett for his help in designing mechanical parts and for their prompt machining, Colin Edwards for keeping all computer systems running, and Angela Campbell for her administration of this project. Thanks to Chris Summerfield, Gary Large, Kevin Mott, and Nigel Palfrey for their machining of the microscope used to conduct all optical work and to Barry Shores and Huw Prytherch for their assistance in designing the integrated electronics. I am grateful to EPSRC for funding this project and to NanoTools GmbH, for their cooperation in providing spherical Au tips for experimentation. Finally, I would like to thank my examiners, Prof.\ Charles Smith and Dr.\ Rupert Oulton, for reading through this work and for their comments and corrections

%Rest of the group
Lastly, my time in Cambridge would not have been the same without the fantastic Robinson College MCR community and the Cambridge University Taekwondo club, in whose company I have spent many hours outside of work. I would especially like to thank Dr.\ Peter Smielewski for everything he has taught me. I would finally like to acknowledge the rest of the Nanophotonics group, specifically those in plasmonics, who have provided many useful discussions and made the group a great place to work over the last four years.

\newpage 
\section*{Abstract}

\begin{raggedright}
{\bfseries\large On the Plasmonic Properties and Dynamic Interactions of Nanostructured AFM Tips}\\[4pt]
{\emph{Alan Sanders}}
\end{raggedright}\\

\noindent
Plasmonics, the confinement of light to nanometric dimensions in the form of optically-driven collective oscillations of conduction electrons, enables strong, local field enhancements, which can be exploited to realise nano-optics and nano-spectroscopy. However, the onset of quantum mechanical effects serves as a fundamental limit to plasmonic confinement in what has recently become known as the quantum regime of plasmonics.

In the present work, a dual AFM tip approach to form sub-nm plasmonic cavities is adopted to investigate the quantum regime and to determine, in particular, the relationship between conductance and plasmonics, a theme of great interest in the field. The technology required to reliably form sub-nm plasmonic cavities between AFM tips is further developed. A custom optical microscope with an ultra-stable nanopositioning platform has been entirely designed and optimised to facilitate experiments. Light scattering from both single and gap-coupled nanostructures can be measured over a broad wavelength range using a novel dark-field spectroscopy technique utilising a supercontinuum white-light laser.

This experimental system has been exploited to fully characterise the optical response of both sharp and spherically nanostructured Au AFM tips in order to understand their plasmonic properties. Additional spherical Au nanoparticle-tipped AFM probes are fabricated using apex-selective pulsed electrodeposition to demonstrate a simple method for introducing localised surface plasmons into a robust tip geometry. Hyperspectral imaging is utilised to optically characterise single nanostructures and identify localised surface plasmons. Spherical Au tips, with their nanoparticle-like apex geometry, are found to exhibit a radiative plasmon resonance between 600--\SI{700}{nm}, not present in sharp Au tips, leading to a 30$\times$ improvement in Raman scattering efficiency compared with sharp Au tips.

Finally, plasmonic interactions between two AFM tips are studied and the transition between coupled and charge transfer plasmons is dynamically observed. Simultaneous measurement of the d.c.\ current, applied force and optical scattering as tips come together is used to determine the effects of a conductive pathway in a plasmonic nano-gap. Critical conductances are experimentally identified for the first time, determining the points at which quantum tunnelling and conductive charge transport begin to influence plasmon coupling. This is a step towards fully understanding the relationship between conduction and plasmonics and the fundamental, quantum mechanical limitations of conventional plasmonic coupling.
%\end{singlespace}

\end{document}