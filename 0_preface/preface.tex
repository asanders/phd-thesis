% !TEX TS-program = pdflatexmk
% !BIB TS-program = bibtex

\documentclass[12pt, a4paper, oneside]{book}
\usepackage{import}
\subimport{../}{preamble}
\ExecuteBibliographyOptions{articletitle=false}
\standalonetrue
\onehalfspacing

\begin{document}
\begin{singlespace}

\section*{Declaration}
The work presented in this thesis was carried out at the Nanophotonics Group in the Cavendish Laboratory, University of Cambridge between October 2011 and June 2015. This dissertation is the result of my own work and includes nothing which is the outcome of work done in collaboration except where specifically indicated in the text. It has not been submitted in whole or in part for any degree at this or any other university, and is less than sixty thousand words long.

{\flushright\emph{Alan Sanders}\par}

\section*{List of Talks, Posters and Publications}
\subsection*{Publications}
\begin{itemize}
\item \fullcite{sanders2014}
\item \fullcite{benz2014}
\item Understanding tip plasmonics
\item Plasmon coupling
\item SERS background
\end{itemize}

\subsection*{Conference Presentations}
\begin{itemize}
\item \fullcite{sanders2015}
\item \fullcite{sanders2014a}
\item \fullcite{sanders2013b}
\item \fullcite{sanders2013a}
\item \fullcite{sanders2012}
\mybibexclude{sanders2015}
\mybibexclude{sanders2014a}
\mybibexclude{sanders2013b}
\mybibexclude{sanders2013a}
\mybibexclude{sanders2012}
\end{itemize}

\newpage
\section*{Acknowledgements}

%I would like to thank...

%Jeremy Baumberg

%Matthew Hawkeye

%Richard Bowman

%Rest of the group

%Technical staff

%Thanks go out to NanoTools Gmbh for their cooperation in providing spherical Au tips for experiments and their assistance tuning the properties of these tips for optical characterisation.

\newpage
\section*{Abstract}

% Introduction and Aims
The recently uncovered quantum regime of plasmonics revealed an interesting new domain of plasmon coupling that necessitates continued investigation. The dual spherical AFM tip approach to forming sub-nm plasmonic cavities is adopted for continuity. Throughout this project, the technology required to initially probe the quantum regime is further developed in order to better study the effects of electron tunnelling on plasmons. To this end, the plasmonic properties of both standard sharp AFM tips and spherically nanostructured AFM tips are studied due to their importance both to the tunnelling regime and in near-field scanning optical microscopy.

% Fabrication of Tips
Both commercially available sharp and spherical Au AFM tips are used in experiments. Additional spherical Au nanoparticle-tipped AFM probes are fabricated using pulsed electrodeposition to demonstrate a simple method for nanostructuring a tip to introduce localised surface plasmons. Using this method in a simple electrochemical cell, single Au nanoparticles are selectively grown onto the apex of a Pt AFM tip. Fabricated spherical Au tips demonstrate increased robustness to damage and resistance to aggressive cleaning treatments, proving useful when requiring sub-nm clean surfaces.
%A more complex cell geometry is used to understand and standardise the growth procedure for a single tip, hence increasing the reliability and reproducibility of the technique.
This provides a novel method for obtaining tips for use in plasmonic applications.

% The Rig
A custom optical microscope is designed and built, capable of measuring light scattering from both single and gap-coupled nanostructures over a broad range of wavelengths using a novel supercontinuum dark-field spectroscopy techinque. By combining an inverted microscope design and supercontinuum white-light laser with an ultra-stable nanopositioning platform, light can be used to optically probe plasmonic nanostructures and the nanometric gaps that define coupled plasmonic systems. Scanning confocal hyperspectral imaging is utilised to determine the scattering response of single nanostructures and identify localised surface plasmons prior to gap coupling experiments.

% Tip Plasmonics
The existence of localised surface plasmons in tips which couple to far-field light is crucial in determining what can be experimentally observed and enables dynamical study of plasmon coupling. Optical spectroscopy shows that sharp metallic tips lack the necessary antenna-like geometry to strongly excite such plasmons. Spherical Au tips are found to support antenna-like plasmons, with a dominant scattering resonance between 600--\SI{700}{nm}. Consequently, excitation of plasmons localised to the spherical tip apex is shown to improve the Raman enhancement by a factor of 30$\times$ compared with sharp Au tips when illuminated on resonance at \SI{633}{nm}.

% Coupling Experiments
Finally, plasmonic interactions between AFM tips are studied. An optomechanical scanning capacitance AFM technique is employed to align tips into a dimer configuration. The separation between two opposing AFM probes is dynamically reduced until geometrical contact. By simultaneously measuring the d.c.\ current, applied force and optical scattering in the quantum regime the effects of an optical conductance on coupled plasmon resonances can be directly inferred. Through this, three critical conductances are dynamically observed/measured for the first time, determining the points at which quantum tunnelling and ballistic transport affect plasmon coupling.

% Sum up impact of work
The presented work is a step forward to fully understanding the effects of conduction on plasmonics and the limitations of classical/conventional plasmonic coupling. Furthermore, characterisation of the local spectral response of tips demonstrates the need to adopt optical characterisation techniques to improve understanding and reliability in tip-enhanced near-field optical microscopy. The improved optical performance of spherically nanostructured tips has implications for the future of tip-based near-field microscopy with the electrochemical production of spherical metallic tips providing a further means to develop improved near-field probes.
\end{singlespace}

\end{document}