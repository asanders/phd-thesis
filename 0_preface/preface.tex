\documentclass{article}
\usepackage{import}
\subimport{../}{preamble}
\begin{document}

\section*{Declaration}
The work presented in this thesis was carried out at the Nanophotonics Group in the Cavendish Laboratory, University of Cambridge between October 2011 and March 2015. This dissertation is the result of my own work and includes nothing which is the outcome of work done in collaboration except where specifically indicated in the text. It has not been submitted in whole or in part for any degree at this or any other university, and is less than sixty thousand words long.

{\flushright\emph{Alan Sanders}\par}

\section*{List of Talks, Posters and Publications}
\subsection*{Publications}
\begin{itemize}
\item \fullcite{sanders2014}
\item \fullcite{benz2014}
\item Understanding tip plasmonics
\item Plasmon coupling
\item SERS background
\end{itemize}

\subsection*{Conference Presentations}
\begin{itemize}
\item \fullcite{sanders2015}
\item \fullcite{sanders2014a}
\item \fullcite{sanders2013b}
\item \fullcite{sanders2013a}
\item \fullcite{sanders2012}
\mybibexclude{sanders2015}
\mybibexclude{sanders2014a}
\mybibexclude{sanders2013b}
\mybibexclude{sanders2013a}
\mybibexclude{sanders2012}
\end{itemize}

\newpage
\section*{Acknowledgements}

I would like to thank...

\newpage
\section*{Abstract}

% Introduction
The plasmonics of both standard sharp AFM tips and spherically nanostructured AFM tips is studied due to their importance in near-field scanning optical microscopy.
% The Rig
To this end a test rig was designed and built, capable of measuring light scattering from both single and gap-coupled nanostructures over a broad range of wavelengths. By combining a custom-built inverted microscope and supercontinuum white-light laser with an ultra-stable nanopositioning platform light can be used to optically probe the nanoscale. Scanning confocal hyperspectral imaging is utilised to determine the scattering response of single nanostructures and identify localised surface plasmons.

% Tip Plasmonics
Results show that sharp Au tips lack the necessary antenna geometry to excite far-field coupleable, localised surface plasmons, whereas spherical tips support such modes. For spherical tips the dominant localised surface plasmon mode is between 600--\SI{700}{nm}. The existence of these modes is important as they dictate what can be seen experimentally.

% Fabrication of Tips
Spherical nanoparticle-tipped AFM probes are electrochemically fabricated to demonstrate a simple method for effectively nanostructuring a tip to induce localised surface plasmon modes. Single AuNPs are grown on the apex of a Pt AFM tip using pulsed electrochemical deposition. Using a simple electrochemical cell geometry, high throughput fabrication of multiple similar AuNP-on-Pt tips is successfully demonstrated. A more complex cell geometry is used to understand and standardise the growth procedure for a single tip, hence increasing the reliability and reproducibility of the technique.

% Brief implications of Spherical Tips
The existence of a plasmon localised to the spherical tip apex is shown to improve the Raman enhancement by a factor of $\sim$30$\times$ when illuminated resonantly at \SI{633}{nm}.
% Coupling Experiments
The interaction between tips is studied. The separation between two opposing AFM probes is dynamically reduced. Scanning capacitance microscopy is employed to align tips into a dimer configuration. Simultaneous measurements of the current through the junction and the applied force are made whilst monitoring the optical response.

% Quantum Effects in Plasmonic Gaps
AuNP-on-Pt tips, unlike commercially available, vacuum-processed spherical tips, are found to be resistant to aggressive cleaning techniques allowing sub-nm gaps to be readily accessed. The plasmonics of sub-nm gaps still remains rich with unknown physics. By monitoring the d.c. conductance in the quantum tunnelling regime the effects of optical conductance can be directly seen. Power transfer between capacitance and conductance plasmon modes is observed once the gap size is reduced by \SI{0.3}{nm} and the d.c. conductance rises above \num{e-3}\G0.

% Sum up impact of work

\end{document}