\documentclass{article}
\usepackage{import}
\subimport{../}{preamble}
\begin{document}

\section{Electrochemical Growth - The Theoretical Background}

Electrodeposition is a method of electrochemistry, the study and application of chemical reactions occurring on the surface of an electrode that are either caused by or generate an electrical current. In general, this requires two electrically connected electrodes, submerged in an ionic solution (a.k.a. electrolyte), with a potential difference. Depending on the electrode potential, and how it compares with the energy required to activate a specific chemical reaction between ionic species, ions are either used or created at the electrode-electrolyte interface through the addition or removal of electrons, respectively. Connecting electrodes enables the transfer of excess electrons donated at one electrode to fill the electron deficit at the other electrode. An electrochemical cell reaction is therefore split into two half reactions to necessitate electron capture and generation processes, with one occurring at each electrode. These processes are known as \textit{oxidation} and \textit{reduction} reactions, which combine to form an overall \textit{redox} reaction. Oxidation occurs at the anode whereby a half reaction is achieved by donating electrons to the electrode. Electrons are then removed from the cathode as positive ions undergo reduction. The rate of both half reactions dictates the rate of electron capture and generation therefore the current flow between electrodes is a measurable representation of the underlying reaction. For this reason electrochemistry is a powerful technique, containing a direct method of directly monitoring the state of a chemical process.

% Discussion of standard potentials
Each oxidation and reaction reaction has an associated standard potential, \gls{E_sp}, characterising the  potential required to add or remove an electron from the chemical species. Chemical species with highly negative standard potentials can readily donate electrons and are known as reducing agents. Chemical species with highly positive standard potentials are more likely to be reduced and are therefore known as oxidising agents. In general, having $E^0<0$ means the chemical species is highly likely to dissociate into an ionic  while $E^0>0$ means a chemical species is more likely to be found in a charge neutral or lesser charged state.

\begin{figure}
\centering
{\small
\begin{subfigure}[t]{0.4\textwidth}
	\def\svgwidth{\textwidth}\subimport{./figures/}{electrochemical_cell.pdf_tex}
	%\caption[Diagram of an electrochemical cell]{\textbf{Diagram of an electrochemical cell.} Two electrodes are submerged in an ionic solution. Depending on the potential, ions can either be oxidised or reduction at each electrode-electrolyte interface, generating a current between electrodes.}
	\label{fig:electrochemical_cell}
\end{subfigure}
~
\begin{subfigure}[t]{0.45\textwidth}
	\def\svgwidth{\textwidth}\subimport{./figures/}{electrochemistry_energy_diagram.pdf_tex}
	%\caption[Energy level diagram of an electrochemical cell]{\textbf{Energy level diagram of an electrochemical cell.} An initial difference in the Fermi levels of the cell, caused by a difference in the work functions of each electrodes, means charge flows to equilibrate the Fermi levels ($1\rightarrow2$). This is the battery effect. The cell can be biased with an applied voltage to initiate electrochemical reactions ($2\rightarrow3$).}
	\label{fig:electrochemistry_energy_diagram}
\end{subfigure}
}
\caption[Schematic and energy level diagrams of an electrochemical cell]{\textbf{Schematic and energy level diagrams of an electrochemical cell.} (a) Diagram of an electrochemical cell, showing two electrodes are submerged in an ionic solution. Depending on the potential, ions can either be oxidised or reduction at each electrode-electrolyte interface, generating a current between electrodes. (b) Energy level diagram of the electrochemical process (right). An initial difference in the Fermi levels of the cell, caused by a difference in the work functions of each electrodes, means charge flows to equilibrate the Fermi levels ($1\rightarrow2$). This is the battery effect. The cell can be biased with an applied voltage to initiate electrochemical reactions ($2\rightarrow3$).}
\end{figure}

% Battery effect
Depending on cell conditions, electrochemistry works in many different ways. In a solution supporting ionic transport with two different material electrodes there is an inbuilt voltage generated by the difference in standard potentials. Electrons move to equilibrate the potential and the resulting current can be used to power an external load. This is the battery effect.%
\footnote{Consider the example of a battery constructed of a Zn cathode and Cu anode in a \ce{Cu2O}/\ce{Zn2O} electrolyte connected via a salt bridge. The half reactions are \cee{Cu^{2+} + 2e- -> Cu} ($E^0=\SI{0.339}{V}$) and \cee{Zn -> Zn^{2+} + 2e-} ($E^0=\SI{-0.76}{V}$) with an overall reaction \cee{Zn + Cu^{2+} -> Cu + Zn^{2+}}. The potential of this reaction is $\phi_{Cu}-\phi_{Zn}=0.339--0.76=\SI{1.099}{V}$.}
% Electroplating
Electrodeposition is an opposite effect caused by applying a voltage across a cell. If both cell half potentials are overcome a redox reaction is activated. Ions from the solution are reduced onto the cathode surface as a complimentary oxidation occurs at the anode. This is the basis of electroplating. Equations representing each half reaction are given by,
\begin{subequations}
\begin{align}
\cee{M+(aq) + e- &-> M(s)}, \mathrm{(cathode)} \label{eq:reduction} \\
\cee{E(s) + e- &-> E^-(aq)}, \mathrm{(anode)} \label{eq:oxidation}
\end{align}
\end{subequations}
where \ce{M} represents a metal and \ce{E} an electrolyte. Metallic ions are reduced on the surface of the cathode whilst the complementary oxidation donates electrons at the anode. The overall reaction can be considered to be,
\begin{equation} \cee{M+(aq) + E^-(aq) -> M(s) + E(s)}, \label{eq:redox} \end{equation}
where the electrodes mediate the reaction. The electrochemical cell voltage of this overall reaction is given by the difference of the half reaction standard potentials,
\begin{equation} E^0_{\mathrm{cell}} = \Eox-\Ered. \end{equation}

\begin{figure}
{\small
\def\svgwidth{0.9\textwidth}
\subimport{./figures/}{electrodeposition_process.pdf_tex}}
\caption[Diagram of the electrodeposition process over time.]{\textbf{Diagram of the electrodeposition process over time.} Metallic ions initially at the interface are completely reduced at short time scales (1). Nearby ions drift to the surface following the field lines. Accumulation of the surface charge at the electrode leads to attraction of a diffuse layer of the oppositely charged ion, partially screening the local field. Depletion of ions in the vicinity of the surface (2) creates a concentration gradient. Further growth is limited by diffusion (3). The balance between drift-limited and diffusion-limited growth is determined by the concentration of the solution.}
\label{fig:electrodeposition_process}
\end{figure}

% The electrodeposition process
The rate at which each half reaction occurs depends on the work functions of the two electrodes and the electric field between them. Work functions dictate which reactions are possible. A potential needs to be applied to correctly bias the electrode to overcome the activation energy and allow a reaction. This allows metallic ions situated at or near the cathode interface to be reduced whilst electrolyte ions on the anode surface oxidise. The accumulated surface charge at the electrode surface (unreacted ions) attracts a more diffuse layer of oppositely charged ions. This is known as the \textit{electrical double layer} and partially screens the local field. After these initial reactions the reaction rate is set by the rate at which either ions or electrons arrive at each interface. The electronic rate depends on the current limit whilst the rate of ion movement is dictated by the electric field and the ion concentration. Nearby ions follow the field lines to the electrode surface where they can react. Depletion of ions used in reactions also leads to the formation of a concentration gradient towards the electrode. Ions then diffuse to these regions. The reaction rate then depends on if the reaction is \textit{diffusion-limited} or \textit{drift-limited}. Each case is determined based on the current ionic velocity components. In analogy with semiconductor physics the current density is given by,
\begin{equation} \vec{J} = q\left(n\mu_i\vec{E} + D\nabla n\right), \end{equation}
where $q$ is the ion charge, $n$ is the number density of charges, \gls{mobility} is the charge mobility and \gls{diffusion_coefficient} is the coefficient of diffusion, therefore $\gls{v_drift} = n\mu_i\vec{E}$ and $\gls{v_diff} = D\nabla n$. Drift depends on the charge carrier concentration and the electric field whereas diffusion depends only on the concentration gradient. Each component can therefore be controlled using the field strength and solution concentration.
% diffusion limited
If the drift velocity is small compared to the diffusion velocity then the reaction is diffusion-limited and can be controlled by varying the concentration of the cell solution.
% field or drift limited
Alternatively, if the diffusion velocity is small compared with the drift velocity, for example in the presence of a large field, then the reaction is field- or drift-limited.
% final limitations
Due to charge conservation the rate of reaction depends on the minimum rate of either the oxidation or reduction reactions. For deposition the minimum rate is ideally set by the rate of reduction.

\subsection{Experimental Electrodeposition of Au}

Electrodeposition is the process used to fabrication AuNP AFM tips. The metallic ions undergo reduction and solidifies while the electrolyte oxidises on the counter-electrode, thus depositing a metallic growth. The dynamics of deposition depend heavily on the potential (electric field), the exposure time and the electrode/cell geometry. Much care has gone into optimising this process to controllably coat an object used as an active electrode in a layer of metal. In this case the technique is typically renamed 'electroplating'.

In experimental electrochemistry a potentiostat is used to apply or measure a potential between two electrodes whilst also measuring the current through the cell. A reference electrode, with a well-known potential, is used to determine the individual potential of the working (active) electrode relative to the solution. %separating measurements of the individual half-reactions.
The simplest reference electrode is the \gls{she}, however many other options exist, including the saturated calomel electrode and the Ag/AgCl electrode. The dissociation of hydrogen into an electron and a proton is set at \SI{0}{V} therefore most electrochemistry methods reference their potentials relative to a \gls{she}, even when using a different reference electrode. A disadvantage to using solution based reference electrodes is their large impedance, which significantly reduces their response time and therefore the temporal bandwidth of potentiostatic measurements. When high bandwidth measurements (high temporal resolution) are required it is necessary to use a Pt reference electrode \cite{sawyer1995electrochemistry}. The large conductivity of the noble metal means the bandwidth is large and Pt is inert to most reactions making it acceptable as a reference point.

There are many available precursor solutions for depositing Au. A simple precursor is \ce{AuCl3} but is not particularly widely used. Many of the available solutions, however, are cyanide based (e.g. \ce{AuCN}), which  pose significant health risks. Sulphite-based solutions (\ce{Au(SO3)2^{3-}}) on the other hand are less hazardous at the cost of deposition quality. One of the most widely used Au precursor electrolytes is the commercial ECF60 solution from Metalor, a safe-use sulphite-based solution. A smooth quality of the deposition morphology is typically achieved by depositing in a low-potential (diffusion-limited) regime and using additives, known as brightener, to achieve smoother coatings. However, when depositing nanostructures, additives are not necessary since the amount of Au deposited is typically very small.
Additionally, ECF60 contains water therefore dilution is simple, giving control of the concentration of Au ions. Its water content also means that a water splitting half reaction occurs at a Pt counter electrode at low potentials to donate electrons to the circuit.
% How does this effect reference measurement

% Potential reactions
Assuming a typical metallic reduction reaction (since the actual composition of ECF60 is proprietary), two sets of possible half cell reactions are \cite{haynes2013crc},
\begin{subequations}
\begin{align}
\cee{2H2O(l) &-> O2(g) + 4H+(aq) + 4e-}, &(\mathrm{anode},\ \Eox&=\SI{1.229}{V}) \label{eq:water_split_ox} \\
\cee{2H+(aq) + 2e- &-> H2(g)}, &(\mathrm{cathode},\ \Ered&=\SI{0}{V})
\end{align}
\end{subequations}
and,
\begin{subequations}
\begin{align}
\cee{2H2O(l) + 2e- &-> H2(g) + 2OH^-(aq)}, &(\mathrm{cathode},\ \Ered&=\SI{-0.8277}{V}) \label{eq:water_split_red} \\
\cee{4OH^-(aq) &-> O2(g) + 2H2O(l) + 4e-}, &(\mathrm{anode},\ \Eox&=\SI{0.401}{V})
\end{align}
\end{subequations}
which both result in a redox reaction of \cee{2H2O(l) -> 2H2(g) + O2(g)}. The standard potential of \eqref{eq:water_split_ox} is $\Eox=\SI{-1.23}{V}$ ($\Ered=\SI{1.23}{V}$ for \eqref{eq:water_split_red}). In general, both these reactions occur simultaneously with a build up of \ce{H+} at the anode and a build up of \ce{OH^-} at the cathode. Electrons produced in this reaction in ECF60 solution also have a chance to reduce Au ions in the solution via (assuming sulphite ions),
\begin{equation} \cee{Au(SO3)2^{3-}(aq) + e- -> Au(s) + 2SO3^{2-}(aq)}. \qquad\qquad (\mathrm{cathode}) \end{equation}
The standard potential for this reaction is $\Ered=0.111$--\SI{0.116}{V} \cite{green2007gold}.
Since this occurs purely at the cathode it competes with the reduction of water. The oxidation of water is then the dominant method of electrolysis leading to a buildup of \ce{H+} in the solution. Growth should then occur on application of a potential \SI{-1.341}{V}, or higher with increased temperature.% should the Nernst potential be taken into account?

\end{document}