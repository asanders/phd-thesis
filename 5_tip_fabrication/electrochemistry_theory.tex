\documentclass{article}
\usepackage{import}
\subimport{../}{preamble}
\begin{document}

\section{Electrochemical Deposition}

%\begin{wrapfigure}{O}{0.4\textwidth}
%\vspace{-10pt}
%\begin{figure}[bt]
%\centering
%{\fontsize{10pt}{1em}\selectfont \def\svgwidth{\textwidth} \subimport{./figures/}{electrochemical_cell.pdf_tex}}
%\caption[Schematic diagram of an electrochemical cell]{\textbf{Schematic diagram of an electrochemical cell.} Two electrodes are submerged in an ionic solution. Depending on the potential, ions can either be oxidised or reduction at each electrode-electrolyte interface, generating a current between electrodes.}
%\label{fig:electrochemical_cell}
%\end{figure}
%\end{wrapfigure}

Electrochemistry is defined as the study and application of chemical reactions occurring on the surface of an electrode that are either caused by or generate an electrical current. Electrochemical growth, or electrodeposition, is a method for depositing a solid material from solution. In general, this requires two electrically connected electrodes, submerged in an ionic solution (electrolyte), with a potential difference. Depending on the electrode potential, and how it compares with the energy required to activate a specific chemical reaction between ionic species, ions are either removed or created at the electrode-electrolyte interface through the addition or removal of electrons, respectively. Connecting electrodes allows excess electrons donated at the anode to transfer and fill the electron deficit at the cathode. An electrochemical cell reaction is therefore split into two half-reactions - electron capture and electron generation. These reactions are known as \emph{oxidation} and \emph{reduction}, which combine to form an overall \emph{redox} reaction. Oxidation occurs at the anode whilst reduction occurs at the cathode. The current flow between electrodes at a given potential therefore measures the underlying reaction rate with electrode potentials setting the range of possible redox reactions. Thus, electrochemistry is a powerful technique, containing an in-built method of directly monitoring the state of the chemical process.

% Discussion of standard potentials
Each oxidation and reduction reaction has an associated standard potential, \gls{E_sp}, characterising the  potential required to add or remove an electron from the chemical species. Chemical species with highly negative standard potentials can readily donate electrons and are known as reducing agents. Chemical species with highly positive standard potentials are more likely to be reduced and are therefore known as oxidising agents. In general, having $\Esp<0$ means the chemical species is highly likely to dissociate into ions, while $\Esp>0$ means a chemical species is more likely to be found in a charge neutral or lesser charged state.
% Battery effect
Electrons move to equilibrate an in-built potential set by the difference between standard potentials at each electrode, $V_{\mathrm{cell}} = \Eox-\Ered$, and the resulting current can be used to power an external load.%
\footnote{Consider the example of a battery constructed of a Zn cathode and Cu anode in a Cu\subs2O/Zn\subs2O electrolyte connected via a salt bridge. The half-reactions are \ce{Cu^{2+} + 2e- -> Cu} ($\Esp=\SI{0.339}{V}$) and \ce{Zn -> Zn^{2+} + 2e-} ($\Esp=\SI{-0.76}{V}$) with an overall reaction \ce{Zn + Cu^{2+} -> Cu + Zn^{2+}}. The potential of this reaction is $\phi_{Cu}-\phi_{Zn}=0.339--0.76=\SI{1.099}{V}$.}
This is the origin of the battery effect.

% Electroplating
Electrodeposition is an opposite effect caused by either applying a voltage across a cell containing metal ions (standard electrodeposition) or by using a reducing agent (electroless deposition) \cite{paunovic2006fundamentals}. If both redox half-potentials are overcome a redox reaction is activated. Metal ions from the solution are reduced onto the cathode surface, via a half-reaction,
\begin{equation} \ce{M^{$n$+}(aq) + $n$e- -> M(s)}, \end{equation}
in which $n$ electrons are reacted with an $n$-oxidation state metal ion as a complimentary oxidation occurs at the anode. This is the basis of electroplating. Only electrode-based electrodeposition is discussed here in the context of electrodepositing onto tips.

\begin{figure}[bt]
{\fontsize{10pt}{1em}\selectfont \def\svgwidth{0.9\textwidth} \subimport{./figures/}{electrodeposition_process.pdf_tex}}
\caption[Diagram of the electrodeposition process over time.]{\textbf{Diagram of the electrodeposition process over time.} Two electrodes are submerged in an ionic solution. Depending on the potential, ions can either be oxidised or reduction at each electrode-electrolyte interface, generating a current between electrodes. Metallic ions initially at the interface are completely reduced on short time scales (1). Nearby ions drift to the surface following the field lines. Accumulation of the surface charge at the electrode leads to attraction of a diffuse layer of the oppositely charged ion, partially screening the local field. Depletion of ions in the vicinity of the surface (2) creates a concentration gradient. Further growth is limited by diffusion (3). The balance between drift-limited and diffusion-limited growth is determined by the concentration of the solution.}
\label{fig:electrodeposition_process}
\end{figure}

% The electrodeposition process
%The rate at which each half-reaction occurs depends on the work functions of the two electrodes and the electric field between them. Work functions dictate which reactions are possible.
A potential difference, $V_{\mathrm{cell}} = \Eox-\Ered$, needs to be applied to correctly bias the cell to overcome the activation energies of half-reactions at each electrode. Metallic ions situated at or near the cathode interface are then reduced whilst electrolyte ions on the anode surface oxidise. The rate of these reactions is set by a number factors, including the rate of charge transfer, nucleation and crystallisation \cite{paunovic2006fundamentals}. This is the point at which reduced ions form adatoms on the surface after charge transfer and must either join with other adatoms to form a critically stable nucleus, crystallise into a specific growth morphology or find an energetically favourable growth location. Excess surface charge accumulated at the electrode surface (unreacted metal ions from the solution and dissociated cathode ions) attracts a more diffuse layer of oppositely charged ions. This is known as the \textit{electrical double layer} and partially screens the local potential, reducing the externally observed field and slowing down the rate of reaction \cite{bard2001electrochemical}.

After the initial surface reactions the rate of reaction becomes influenced by mass transport - the arrival of either ions or electrons at each electrode interface. The rate of ion movement towards the interface can be categorised into three types of mass transport: diffusion, migration and convection. The electric field and electrophoretic mobility determines the migration (drift) current whereas local ion concentration sets the diffusion flow. Convection occurs if there is a thermal gradient in the cell. The main effects considered in this work as the dominant mass transport mechanisms are diffusion and migration.
During mass transport, nearby ions follow a trajectory to the electrode surface set by the field lines of the cell potentials and the concentration gradients pointing to locations of high electrochemical activity and local ion depletion. The extent of the ion depletion region around a growth is known as its diffusion zone and becomes important when two expanding diffusion zones overlap. Under these situations mass transport becomes divided between two competing growth sites, stunting growth. The overall rate of an electrochemical reaction (its rate determining step) will be limited by whichever of the previously discussed effects is slowest since each process occurs in series.

% Underpotential vs. Overpotential growth
Electrodeposition is often described in terms of an \emph{overpotential}, $\eta = E_{\mathrm{applied}} - \Esp$, that is the difference between the in-built potential of an electrolytic solution and the applied potential across it.  In this case, \Esp\ is defined for the reduction of a metal ion onto an electrode of the same metal. Deposition of this metal will only occur once the applied potential is more negative than \Esp, i.e. $\eta<0$. This describes the potential at which enough adatoms form critically stable nuclei on the metal surface. The more negative the overpotential the faster charge transfer, nucleation and drift will occur and therefore the faster the reaction will occur. However, favourable interactions mean that some metals deposit easier onto electrodes made of other materials. Reduced adatoms are able to find energetically favourable locations to grow without the need to form critical nuclei. In this case, deposition can begin even if $\eta>0$. This is known as \emph{underpotential} deposition and is useful to consider if a slow, controlled growth is desired.

\subsection{Experimental Electrodeposition of Au}

% Schematic of reference electrode usage?

Electrodeposition is the process used to fabrication spherical Au \gls{afm} tips by exploiting the sharp profile of the tip apex. The dynamics of deposition depend heavily on the potential (electric field), the exposure time and the electrode/cell geometry and solution. In experimental electrochemistry, a potentiostat is used to apply or measure a controlled potential between two electrodes and the solution whilst also measuring the current through the cell. A reference electrode, with a well-known potential, is used to determine the individual potential of the working (active) electrode relative to the solution. The most fundamental reference electrode is the \gls{she} since the dissociated of hydrogen is defined as $\Esp=\SI{0}{V}$ \cite{paunovic2006fundamentals}, however, due to its complex setup, many other alternatives exist, including the saturated calomel electrode and the Ag/AgCl electrode. Potentials referenced using these other electrodes are often quoted relative to a \gls{she} using the known reference potential. In the simplest of cells, reference electrodes can be in principal ignored and the counter electrode is used as a potential reference point.

Au can be deposited using one of the many available precursor solutions. A simple precursor is AuCl\subs3 but it is not particularly widely used. A large number of solutions are cyanide based (e.g. \ce{AuCN}), which pose significant health risks. Sulphite-based solutions (\ce{Au(SO3)2^{3-}}) on the other hand are less hazardous at the cost of deposition quality. One of the most widely used sulphite-based Au precursors is the commercial ECF60 solution from Metalor. A smooth quality of the deposition morphology is typically achieved by slowly depositing in a low-potential (diffusion-limited) regime and using additives, such as brightener, to achieve smoother coatings \cite{oniciu1991some}. However, when depositing nanostructures, additives are not necessary since the amount of Au deposited is typically very small. Additionally, ECF60 is water soluble \cite{roy2009electrochemical}, simplifying dilution for control over the Au ion concentration.

% Potential reactions
Since the actual composition of ECF60 is proprietary it's exact redox reaction is unknown. Known information includes it being a sulphite-based Au precursor consisting of \SI{0.05}{M} Au and \SI{0.24}{M} Na\subs2SO\subs3 at a pH of 9.5 \cite{roy2009electrochemical}. The expected reduction reaction for a sulphite-based Au precursor is \cite{green2007gold},
\begin{equation}
	\ce{Au(SO3)2^{3-}(aq) + e- -> Au(s) + 2SO3^{2-}(aq)}. \qquad (\mathrm{cathode},\ \Ered=0.111\textendash\SI{0.116}{V}).
\end{equation}
The redox reaction is expected to be satisfied by oxidation of the SO\subs3\sups{2-} species, however complementary water splitting reactions are also expected at higher potentials. Two sets of half-reactions for the water splitting are \cite{haynes2013crc},
\begin{subequations}
\begin{align}
	\ce{2H2O(l) &-> O2(g) + 4H+(aq) + 4e-}, &(\mathrm{anode},\ \Eox&=\SI{1.229}{V}) \label{eq:water_split_ox} \\
	\ce{2H+(aq) + 2e- &-> H2(g)}, &(\mathrm{cathode},\ \Ered&=\SI{0}{V})
\end{align}
\end{subequations}
and,
\begin{subequations}
\begin{align}
	\ce{2H2O(l) + 2e- &-> H2(g) + 2OH^-(aq)}, &(\mathrm{cathode},\ \Ered&=\SI{-0.8277}{V}) \label{eq:water_split_red} \\
	\ce{4OH^-(aq) &-> O2(g) + 2H2O(l) + 4e-}, &(\mathrm{anode},\ \Eox&=\SI{0.401}{V})
\end{align}
\end{subequations}
both resulting in a redox reaction of \ce{2H2O(l) -> 2H2(g) + O2(g)}. The standard potential of \eqref{eq:water_split_ox} is $\Eox=\SI{-1.23}{V}$ ($\Ered=\SI{1.23}{V}$ for \eqref{eq:water_split_red}). In general, both these reactions occur simultaneously with a build up of H\sups+ at the anode and a build up of OH\sups{\textminus} at the cathode. %The reduction of Au inevitably competes with the reduction of water.
Using this solution, significant growth occurs once $V<\SI{-0.6}{V}$, as has been experimentally determined, with an increase in the number of reactions once $V<\SI{-1.23}{V}$.

\end{document}