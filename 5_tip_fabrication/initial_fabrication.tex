\documentclass{article}
\usepackage{import}
\subimport{../}{preamble}
\begin{document}

\section{Initial Fabrication of Spherical AuNP-Tipped AFM Probes}
\label{sec:initial_fabrication}

\begin{figure}[bt]
\centering
\small
\def\svgwidth{0.6\textwidth}
\subimport{./figures/}{initial_setup.pdf_tex}
\caption[Experiment geometry for pulsed electrochemical deposition of Au onto an AFM tip]{\textbf{Experiment geometry for pulsed electrochemical deposition of Au onto an AFM tip.} Electrochemical cell for growth of Au onto the apex of an AFM tip. Termination of field lines at tip apex due to the lightning rod effect enhances localised electrochemical growth for single NP growth.}
\label{fig:initial_setup}
\end{figure}

A simple initial method is used to successfully demonstrate that a spherical AuNP can be electrochemically grown at the apex of a metallic AFM probe tip.
% Initial method and setup description
Fabrication of spherical AuNP tips onto commercial AFM probes (BudgetSensors GB/E series) is achieved using single-pulse high-field electrochemical growth. Conductive coatings are required for the electrochemical reaction, therefore Au- and Pt-coated AFM tips are used. The adhesion of molecular layers can prevent deposition of Au therefore tips must be cleaned thoroughly prior to deposition. Tips are pre-treated with\SI{20}{\minute} oxygen plasma to remove organic contaminants from the surface prior to growth. A simplified two-electrode system (AutoLab PGSTAT 302N potentiostat) is employed for growth since both the cell geometry and electrodeposition solution are kept the same between fabrications. AFM probes are attached to fluorine-doped tin oxide (FTO) conductive glass, used as a working electrode, opposite a Pt wire counter-electrode, spaced \SI{10}{mm} apart (\figurename~\ref{fig:initial_setup}). FTO glass is cleaned through sonication in \SI{10}{\minute} steps using de-ionised water, ethanol and finally acetone. Metalor ECF60 is used as the electroplating solution with no additives. Simultaneous fabrication of both Au and Pt tips is carried out by contacting multiple tips, closely spaced side-by-side on the same FTO surface.

\begin{figure}[bt]
\centering
\includegraphics{figures/initial_cv}
\caption[Linear sweep voltammetry of a single AFM cantilever in ECF60]{\textbf{Linear sweep voltammetry of a single AFM cantilever in ECF60.} The AFM probe is held by an Al clamp cathode out of solution, replacing the standard FTO electrode to show growth characteristics of only the tip. The remaining geometry is the same as in \figurename~\ref{fig:initial_setup}.}
\label{fig:initial_cv}
\end{figure}

Linear sweep voltammetry of a single AFM tip in ECF60 (\figurename~\ref{fig:initial_cv}) shows that Au growth starts at \SI{-1.1}{V}. This is similar to the predicted reaction potential for the reduction of Au and the oxidation of water. The increased potential is expected due to working at room temperature. % is this strictly true? Nernst potential

A single high-voltage pulse is applied to nucleate and grow a single AuNP at the tip apex. Due to the large field amplitude at the tip apex, field lines from across the cell terminate at the apex inducing ions to drift towards the tip (Figure \ref{fig:electrochemistry_setup}b).  Multiple combinations of applied voltage and pulse time were investigated to optimise growth parameters. The growth of Au onto the AFM tip is confirmed by current dynamics, revealing a 2--\SI{3}{ms} initiation followed by relaxation to continuous diffusion-limited growth within a few 10s of milliseconds. SEM imaging, carried out on a LEO GEMINI 1530VP FEG-SEM Scanning Electron Microscope, is used to characterise the resulting growth morphology.

\begin{figure}[bt]
\centering
\includegraphics[width=0.9\textwidth]{figures/figure_2}
\caption[Comparison of AuNP-tipped AFM probes, fabricated on various base structures using \SI{-8}{V} pulses of different lengths]{\textbf{Comparison of AuNP-tipped AFM probes, fabricated on various base structures using \SI{-8}{V} pulses of different lengths.} The first three tips were produced simultaneously using a \SI{200}{ms} pulse on (a) a commercial Pt tip with no pre-treatment, (b) a plasma-treated Pt tip, (c) a plasma-treated Au tip. (d) Duplicate spherical tip produced separately on a plasma treated Pt tip using a \SI{100}{ms} pulse.}
\label{fig:electrochemical_tips}
\end{figure}

SEM images of four tip samples fabricated at \SI{-8}{V} are shown in \figurename~\ref{fig:electrochemical_tips}, three of which were fabricated simultaneously on one FTO glass slide, to show the effects of tip pre-treatment and exposure time in the applied field. These images demonstrate that spherical AuNP tips can be reliably fabricated using the proposed electrodeposition procedure. Spherical AuNP growth diameters between 150--\SI{450}{nm} are achieved using 100--\SI{200}{ms} pulses across the electrochemical cell on both Pt and Au tips. Evidence of growth localisation to the high field regions is clearly exhibited by the formation of spherical AuNPs at the tip apex. The morphologies obtained differ with and without plasma pre-treatment. Using tips as supplied leads to a smooth sphere at the tip apex resulting from the lightning rod effect, followed by a broad neck and semi-uniform Au coating across the exposed surfaces of the tip (\figurename~\ref{fig:electrochemical_tips}a). Plasma treatment removes organic contamination and can also oxidise the surface \cite{li2003, fuchs2009}. Removal of contaminants prevents growth at defect sites. The formation of an insulating metal oxide layer prevents growth on surfaces, limiting growth only to sharp regions with a small radius of curvature (\figurename~\ref{fig:electrochemical_tips}b--d). These regions remain conductive and the electric field is large and highly localised. Because Au is more difficult to oxidise than Pt, the shielding effect is different for plasma-treated Au-coated AFM probes, which exhibit significant localised nanoparticle growth on all exposed surfaces (\figurename~\ref{fig:electrochemical_tips}c). A longer pulse time can result in a bigger diameter of spherical AuNP, as shown in \figurename~\ref{fig:electrochemical_tips}b and \figurename~\ref{fig:electrochemical_tips}d. A 200 ms pulse growth leads to a spherical AuNP on the Pt tip with a diameter of $\sim$\SI{450}{nm} (\figurename~\ref{fig:electrochemical_tips}b), while the diameter of the AuNP is \SI{150}{nm} with a \SI{100}{ms} pulse (\figurename~\ref{fig:electrochemical_tips}d).

%Morphology Dependence
\begin{figure}[bt]
\centering
\includegraphics{figures/tip_voltage_dependence}\\
\includegraphics{figures/current_transients}
\caption[Voltage dependence of pulsed electrodeposition onto Pt tips]{\textbf{Voltage dependence of pulsed electrodeposition onto Pt tips.} Comparison between a standard Pt AFM tip (a) and AuNP growth on plasma-treated Pt tips (b-f) using \SI{150}{ms} pulses for voltages between \num{-8} and \SI{-3}{V} and a \SI{500}{ms} pulse at \SI{-2}{V}. This shows the change in deposition mechanism as the magnitude of applied field strength is increased, with no spherical growths above \SI{-3}{V} irrespective of exposure time. (g) Current transients from current traces measured during fabrication of tips shown in (b-f), offset by the saturation current density (\num{-10}, \num{-31}, \num{-34}, \num{-78}, \SI{-142}{mA\per cm\squared}, respectively). (h) Variable-independent reduced current transients (coloured lines) measured at various applied voltages during fabrication compared with theoretical curves for progressive (dashed) and instantaneous (solid) nucleation current transients \cite{scharifker1983}.}
\label{fig:electrochemical_voltage_dependence}
\end{figure}

To investigate the dependence of fabricated tip morphology on the pulsed voltage across the electrochemical cell, the growth of AuNP on Pt tips at different applied voltages is further studied. SEM images of such fabrications along with corresponding current transients are shown in Figure \ref{fig:electrochemical_voltage_dependence}. Images show that apex-selected growth occurs only once the voltage is more negative than \SI{-3}{V}. This voltage dependence is attributed to changes in deposition mechanism with field strength.
At low voltages ($\geq \SI{-2}{V}$), electrodeposition forms smooth film coatings (Figure \ref{fig:electrochemical_voltage_dependence}b) similar to direct-current electrodeposition. Under these conditions growth is dominant over nucleation and the field profile caused by the lightning rod effect is eventually evened out, yielding smooth rounded tips. Even with \SI{500}{ms} exposure time no apex-selective growth is observed despite the charge transfer being equivalent to AuNP tip growths at more negative potentials.

%Electrochemical Nucleation Processes
%\begin{wrapfigure}{O}{0.55\textwidth}
\begin{figure}[bt]
\centering
%\vspace{-10pt}
{\includegraphics{figures/nucleation_theory}}
{\caption[Reduced current transients for the two extremes of nucleation \cite{sharifker1983}.]{\textbf{Reduced current transients for the two extremes of nucleation \cite{sharifker1983}.} The normalised current, $(i/i_m)^2$, is plotted as a function of time normalised to the maximum, $t/t_m$. Dashed lines show reduced transients from linear transitions between instantaneous nucleation, assuming instantaneous nucleation occurs first then decreases.}
\label{fig:nucleation_theory}}
%\vspace{-10pt}
\vspace{-5pt}
%\end{wrapfigure}
\end{figure}

The electrodeposition morphologies on tips fabricated at potentials more negative than \SI{-2}{V} can be understood to a certain degree by considering the most common nucleation mechanisms known from AuNP growth on planar electrodes \cite{scharifker1983}. These are progressive and instantaneous nucleation. During progressive nucleation, nuclei form at a time $t$ then grow, increasing the size of the diffusion zone around them, thereby preventing further nucleation within that zone. At the end of progressive nucleation there are many nuclei of different sizes due to the different times at which they originally nucleated. During instantaneous nucleation all nuclei are nucleated at $t=t_0$ and then grow at the same rate until their diffusion zones overlap. The overlap of diffusion zones stunts growth as the finite ion flow is split between the two nuclei. For a sparse distribution of nuclei the end result is ideally a set of equally sized particles. The current transients for progressive and instantaneous nucleation are given in the Scharifker and Hills model by,
\begin{equation}
\left(\frac{i}{i_{\mathrm{max}}}\right)^2 = \frac{1.2254}{t/t_{\mathrm{max}}} \left[ 1 - e^{-2.3367(t/t_{\mathrm{max}})^2} \right]^2,
\label{eq:prog_nucleation}
\end{equation}
and.
\begin{equation}
\left(\frac{i}{i_{\mathrm{max}}}\right)^2 = \frac{1.9542}{t/t_{\mathrm{max}}} \left[ 1 - e^{-1.2564(t/t_{\mathrm{max}})} \right]^2,
\label{eq:inst_nucleation}
\end{equation}
respectively, where $t_{\mathrm{max}}$ is the time at which the current density maximises at $i_{\mathrm{max}}$.

% Applying nucleation mechanisms to explain the SEMs
Increased nucleation is observed at \SI{-3}{V} where spherical tip growth is initiated under a progressive nucleation mechanism with some preferential growth at the tip apex, as evidenced by the large number of nucleated particles and variation in particle size (Figure \ref{fig:electrochemical_voltage_dependence}c). For more negative voltages ($\leq \SI{-4}{V}$) nucleation becomes more selective, leading to clean surfaces and improved apex-selected growth (Figure \ref{fig:electrochemical_voltage_dependence}d-f). This is due to a transition to instantaneous nucleation, in which a fixed number of particles nucleate at selected active sites on application of a field \cite{hyde2003}. This occurs preferentially at sharp edges where the field is highest. Further selection may occur through depletion of ions in the vicinity of the growing tip, preventing additional growth sites. This helps to produce an isolated AuNP at the tip apex. Increasing the magnitude of the voltage increases the number of active sites available for nucleation and more of the tip surface surpasses the field threshold for instantaneous nucleation. Hence, using less negative voltages within the instantaneous nucleation regime reduces the number of active nucleation sites leading to cleaner spherical growth at the tip apex.

% Current transients
This changeover in nucleation mechanisms is also observed in current transients (\figurename~\ref{fig:electrochemical_voltage_dependence}g)%
\footnote{Current measurements are interpolated to create a smooth line through data points. Transients are analysed by first making a quadratic interpolation of experimental data points before data reduction to extract the reduced current transients. This approach is used due to the limited number of points available in the peak region.}
as the shape distinctly changes when decreasing the voltage below \SI{-2}{V}. The largest change in transient shape occurs at \SI{-3}{V}, indicating the onset of short time-scale nucleation. Elongation of the transient time is likely caused by contribution to the current from progressive particle nucleation throughout the exposure. The large short-time-scale transients observed for potentials more negative than \SI{-4}{V} support the hypothesis of an instantaneous nucleation mechanism as the fast current decay indicates the saturation of all active sites leaving only diffusion-limited growth.

The influence of instantaneous nucleation for isolated AuNP tip growth is further evident in comparisons to theoretical reduced current transients for diffusion-limited progressive and instantaneous nucleation (Figure \ref{fig:electrochemical_voltage_dependence}h) developed by Scharifker and Hills (SH model)\cite{scharifker1983}. Reduced current transients are normalised to $(i/i_{\mathrm{max}})^2$ and plotted against $t/t_{\mathrm{max}}$ to remove variable dependencies, where $i_{\mathrm{max}}$ and $t_{\mathrm{max}}$ represent the peak current and corresponding time. In general, for less negative deposition potentials (\SI{-2}{V}), nucleation resembles more closely progressive nucleation and growth, while at more negative potentials ($< \SI{-4}{V}$) it resembles more closely instantaneous nucleation. This correlates well with the SEM images shown in Figure \ref{fig:electrochemical_voltage_dependence}d-f. Variations from theory occur due to the variable field profile present across the tip leading to localised instances of both progressive and instantaneous nucleation contributing to the overall current. The SH model is built upon assumptions such as random nucleation and provides valid results only for the limiting cases of only progressive or instantaneous nucleation for which the validity fails in such a selectively grown system\cite{dudin2010}.

% More in-depth explanation and model
A simple model can be developed using the SEMs and current transients. Whilst the large surface area of the FTO glass likely dominates the behaviour of the current density, SEM images do show localised changes in the morphology of the tip. Identification of the nucleation mechanism through the reduced current transients is likely only applicable to the FTO surface but does give some information regarding the apex behaviour. When the apex of the tip has a field profile above the threshold for instantaneous nucleation, caused by the increase in field due to the lightning rod effect, a single particle may nucleate at the apex and quickly grow. As it nucleates quicker than the rest of the tip it's diffusion zone increases to prevent particles nucleating and growing nearby. This leads to a clean single AuNP growth at the tip apex. If the threshold for instantaneous nucleation is not met then many smaller particles form around the apex (\figurename~\ref{fig:electrochemical_voltage_dependence}c). On the other hand if too much of the tip can nucleate instantaneously then the chance for a clean growth is again reduced (\figurename~\ref{fig:electrochemical_voltage_dependence}f) since many particles nucleate around the apex. Through this reasoning it is apparent that the choice of voltage is imperative to enable selective growth of a single AuNP at the apex.

\end{document}